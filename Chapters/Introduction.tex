\lhead{\emph{Introduction}}
\chapter{Introduction}
Remote communications have become the backbone for industries at sea. The energy and shipping industries remote sites are, in the case of fixed position vessels, are remote or, in the case of ships at sea, traveling long distances. The communication infrastructure used by these remotes has the defining characteristics of low bandwidth, high latency, and intermittent connectivity. Under these conditions, there is demand for secure capture and transmission of information.\cite{Levinson:2010wm} This project serves to present a system, which can securely deliver that data. 
This paper presents a high level overview of the system; it’s architecture, the component systems, and a reference implementation. Upon that, a threat analysis is used as a benchmark. Several attacks are employed using both classical and modern techniques as of this writing.
The research presented here are intended to address a problem in securing that information. The benefits to this research are to make this data available for analysis and mitigate the potential for loss. We do this by describing methods to capture and, securely transmit the data. Securing the information assures that it is free from corruption, unauthorized access and from a confirmed origin.
