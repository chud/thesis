\lhead{\emph{Problem Definition}}
\chapter{Problem Definition} 
\section{Motivation and Sample Applications}
The wide geography covered by the vessels requires global coverage. Because of this few assumptions can be made about the network link from region to region. 
The problem of remote data security exists in many use case applications. In this paper we reference example applications involving maritime shipping, and maritime energy production. Maritime, as it refers to shipping, has challenges regarding the tracking and monitoring of cargo. Worldwide, there are in excess of seventeen million shipping containers.\cite{Levinson:2010wm} They are used to transport perishable cargo or raw materials otherwise sensitive to environmental attributes such as temperature, humidity, vibration or tilt. 
A specific example might be a shipping of New England tuna bound for Japan. In this case, proposed shipping requirements in the fishing industry require that the temperature must strictly maintain a temperature of -40° C to –45° C during transit. A solution must allow for logging values and, alarming for environment thresholds (temperature, humidity), and security (doors opened in transit) in those containers. need to be stored until a transmission path becomes available.
Petroleum Energy production has seen a drastic increase in production since in recent years as the price of crude has stabilized at historic highs.\cite{Anonymous:ko4JQlUG} Drilling efforts are moving into more complex environments, such as offshore drilling, and deep water drilling in the Gulf of Mexico. By virtue of the terrain surrounding deep-water environments, terrestrial communications infrastructure is not available. The most common solution is a satellite connection. Operating details such as those from the drill, pumps, storage tanks and real-time activities require continuous data communications with systems on shore. A specific case can be drawn from the explosion aboard the Drill Ship Deepwater Horizon. After the explosion on April 20th 2010, a preservation of evidence order \cite{Fenton:2011kq}was announced. Our system can potentially create and maintain security of that collected evidence and this can be done even before a triggering event occurs (e.g., data collected immediately before the explosion). In the Gulf of Mexico there are 2400 drilling rigs of which two-dozen are operating in deep water\cite{Anonymous:ko4JQlUG}. The potential need for secure information has become a necessity. To be secured, the properties authentication, availability, confidentiality and origin integrity.)
Tracking of air pollution data at relay stations where the security of such data must follow a chain of ownership.  The evidence gathered in a crime scene where a chain of customer must be maintained.\cite{Group:2006en}  The processing of event data in a drilling site where timestamps and irrefutably are crucial for investigation when that remote may no longer have the data stored on the remote. 

\section{Related Work}
Palsberg et al, \cite{Palsberg:1997ts} discuss an end to end solution for transmission of messages. The solution they present does inform our decisions however it lacks a provision for handling situations when the transmission link is unavailable. We address this through the design of a store-and-forward solution.
Schneier \cite{Schneier:1999fu} proposes a method for securing log entries. The solution requires the use of a trusted third party. We address this with a pre-shared, out-of-band distribution of certificates.
A useful work from Mumar and Handzinski\cite{Kumar:2012vg} provides details on how RESTful protocols can be applied. We reference this in the implementation as background for are use of REST.
\section{Assumptions}
This solution assumes that the physical access to the remote device is controlled and the data inputs are properly managed. Specifically, the collection devices and their inputs outside of the system are considered to be secure. For the system these devices and inputs are taken to be the ingress point for raw data. It is at this point that we consider the information secure. The data from here is passed via messages that are defined below. Equally, the processing and storage hardware at the data center is secured assumed to be physically secure. Access to persistent storage (hard disk or SSD), memory, or the network interface adapter, is a significant vulnerability and outside of the scope of this investigation. 
The keys used in the communications are provided to the remote device in a secure manner before installation. The presence of this pre-existing key agreement is delivered out-of-band of the network path used by the messaging and loaded onto the node at install time. 
The sites’ inability to rely on high bandwidth, low latency associated with terrestrial communication links provides challenges. Our system assumes and addresses the need to secure the data while the data center server is unavailable. During the periods in which communications are not available, the nodes will secure and store the messages until connectivity is resumed.
     The use of the term security in this paper is used with respect to confidentiality, integrity, availability, authenticity, and non-repudiation. These security properties and maintained throughout the system.
     
\section{Information Security Attributes Definitions}
Security and the properties of security are listed here for clarity. These definitions represent the terms usages in the paper.
\subsection{Integrity}
The integrity in our system refers to "\ldots the trustworthiness of data or resources."
\cite{Bishop:2003ug} Integrity prevents the undetected change to the data of a message. Message integrity applies to the message, both when it is being transmitted over the network (‘in-flight’)  as well as when it is stored on persistent storage and when it is in memory.\cite{Levinson:2010wm}
\subsection{Confidentiality}
Confidentiality is the property that prevents unauthorized users from being able to read the message. Encrypting the data portion of the message with a high-entropy key and secure algorithm creates a message that an observer cannot decipher. The system must maintain confidentiality of the messages through the system including when stored. We provide this using multiple mechanisms in the message’s lifecycle. 
\subsection{Authentication (Origin Integrity)}
Origin Integrity is the property whereas a sender cannot be impersonated. todo 


[There exist different types of ‘origin integrity’: creator OI, signer OI, sender OI, etc. Are you planning on dealing with all of them?]


\subsection{Availability} 
Availability is the most critical concern of this system. As messages are generated onboard the remote site, critical data is being created. Each message, representative of an event, is part of the overall status of that site. In our problem we focus on the assumption that the backhaul to the data center is intermittent. In order to ensure the availability of the data we secure store the messages until the backhaul returns and connectivity is reestablished. Availability, in this strict sense, is the presence of the messages to be transmitted.
\subsection{Non-Repudiation}
Non-repudiation is the … “security service by which the entities involved in a communication cannot deny having participated.”\cite{Kissel:2013vu} In our particular case we assert that a message having been sent cannot be denied by the sender. This is of great importance in our solution. Use cases where the messages leading up to an event must be veritably sent from a known sender (or senders). The two components of non-repudiation are message integrity and origin integrity (authentication).  Message integrity is provided using the hash message authentication code (HMAC). Origin integrity is provided using a digital signature. 

